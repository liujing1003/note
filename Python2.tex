\documentclass{article}
\usepackage{CJKutf8}
\usepackage{minted}
\begin{document}
\begin{CJK}{UTF8}{gbsn}\title{Python}
\date{}
\maketitle
\section{什么是函数}
\begin{minted}{python}
# 函数调用s=area_of_circle(x)
\end{minted}
\section{调用函数}
\paragraph{}
要调用一个函数,需要知道函数的名称和参数
\paragraph{}
调用abs函数
\begin{minted}{python}
abs(-20)
\end{minted}
\paragraph{}
如果函数传入参数数量不是一个或格式不正确都会报错。
\paragraph{}
而比较函数cmp(x,y)就需要两个参数,如果x<y,返回-1,如果x==y,返回0,如果x>y,返回1
\paragraph{}
Python的内置常用函数还包括数据类型转换函数,比如int()函数可以把其他数据类型转换为整数,str()可以把其他类型转换成str
\section{编写函数}
\paragraph{}
定义一个函数要用def语句,依次写出函数名,括号,括号中的参数和冒号:,然后在缩进块中编写函数体,函数的返回值用return语句返回。
\begin{minted}{python}
#我们以自定义一个求绝对值的my_abs函数为例:
\end{minted}
\begin{minted}{Python}
def my_abs(x):
   if x>=0:
      return x
   else:
      return -x
\end{minted}
\paragraph{}
请注意,函数体内部的语句在执行时,一旦执行到return时,函数就执行完毕,并将结果返回。因此,函数内部通过条件判断和循环可以实现非常复杂的逻辑。
\section{返回多值}
\begin{minted}{python}
# #math包提供了sin()和cos()函数,我们先用import引用它
import math
def move(x,y,step,angle):
   nx=x+step*math.cos(angle)
   ny=y-step*math.sin(angle)
   return nx,ny
\end{minted}
\paragraph{}
Python函数返回的仍然是单一值,是一个tuple,在语法上,返回一个tuple可以省略括号,而多个变量可以同时接受一个tuple,按位置赋给对应的值,所以,Pathon的函数返回多值其实就是返回一个tuple,但写起来更方便。
\section{递归函数}
\paragraph{}
在函数内部,可以调用其它函数。如果一个函数在内部调用自身本身,这个函数就是递归函数。
\begin{minted}{python}
# fact(n)用递归的方式写出来就是:
def fact(n):
   if n==1:
      return 1
   return n*fact(n-1)
\end{minted}
\section{定义默认参数}
\paragraph{}
定义函数的时候,还可以有默认参数。
\paragraph{}
例如python自带的int()函数,其实就有两个参数,我们既可以传一个参数,又可以传两个参数,int的第二个参数是转换进制,如果不传,默认是十进制。
\paragraph{}
我们来定义一个计算x的N次方的函数:
\begin{minted}{python}
def power(x,n):
    s=1
    while n>0:
        n=n-1
        s=s*x
    return s
\end{minted}
\paragraph{}
由于函数的参数按从左到右的顺序匹配,所以默认参数只能定义在必须参数的后面。
\section{可变参数}
\paragraph{}
如果想让一个函数能接受任意个参数,我们就可以定义一个可变参数:
\begin{minted}{python}
def fn(*args):
    print args
\end{minted}
\paragraph{}
可变参数的名字前面有个*号,我们可以传入0个,1个或多个参数给可变参数:
\begin{minted}{python}
fn()
()
fn('a')
('a',)
fn('a','b')
('a','b')
fn('a','b','c')
('a','b','c')
\end{minted}
\paragraph{}
Python解释器会把传入的一组参数组装成一个tuple传递给可变参数,因此,在函数内部,直接把变量args看成一个tuple就好了。
\section{对list进行切片}
\paragraph{}
取前三个元素,用一行代码就可以完成切片:
\begin{minted}{python}
L[0:3]
#表示从索引0开始取,直到索引3为止。但不包括索引3.即索引0,1,2,正好3个元素。
#如果第一个索引是0,还可以省略
L[:3]
#也可以从索引1开始,取出2个元素出来
L[1:3]
#只用一个:,表示从头到尾
L[:]
#因此,L[:]实际上复制出了一个新list
#切片操作还可以指定第三个参数
L[::2]
L = ['Adam', 'Lisa', 'Bart', 'Paul']
>>> L[::2]
['Adam', 'Bart']
#第三个参数表示每N个取一个,上面的L[::2]会每两个元素取出一个来,也就是隔一个取一个。
#把list换成tuple,切片操作完全相同,只是切片的结果也变成了tuple。
\end{minted}
\paragraph{}
\begin{minted}{python}
#range()函数可以创建一个数列
range(1,101)
[1,2,3,...,100]
\end{minted}
\section{倒序切片}
\paragraph{}
\begin{minted}{python}
L = ['Adam', 'Lisa', 'Bart', 'Paul']

>>> L[-2:]
['Bart', 'Paul']

>>> L[:-2]
['Adam', 'Lisa']

>>> L[-3:-1]
['Lisa', 'Bart']

>>> L[-4:-1:2]
['Adam', 'Bart']
#记住倒数第一个元素的索引是-1.倒叙切片包含起始索引,不包含结束索引。
\end{minted}
\section{对字符串切片}
\paragraph{}
\begin{minted}{python}
>>> 'ABCDEFG'[:3]
'ABC'
>>> 'ABCDEFG'[-3:]
'EFG'
>>> 'ABCDEFG'[::2]
'ACEG'
\end{minted}
\paragraph{}
字符串有个方法upper()可以把字符变成大写字母
\begin{minted}{python}
>>> 'abc'.upper()
'ABC'
\end{minted}
\section{什么是迭代}
\paragraph{}
迭代就是对于一个集合,无论该集合是有序还是无序,我们用for循环总是可以依次取出集合的每一个元素。
\paragraph{}
集合是指包含一组元素的数据结构,我们已经介绍的包括:
\\1.有序集合:list,tuple,str和unicode;
\\2.无序集合:set
\\3.无序集合并且具有key-value对:dict
\paragraph{}
迭代是一个动词,它指的是一种操作,在python中,就是for循环。
\section{索引迭代}
\paragraph{}
Python中,迭代永远是取出元素本身,而非元素的索引。我们想要在for循环中拿出索引,要使用enumerate()函数:
\begin{minted}{Python}
>>> L = ['Adam', 'Lisa', 'Bart', 'Paul']
>>> for index, name in enumerate(L):
...     print index, '-', name
... 
0 - Adam
1 - Lisa
2 - Bart
3 - Paul
\end{minted}
\paragraph{}
使用enumerate()函数,我们可以在for循环中同时绑定索引index和元素name。但是,他实际相当于将迭代的每一个元素变成了一个tuple:
\begin{minted}{python}
for t in enumerate(L):
    index = t[0]
    name = t[1]
    print index, '-', name
\end{minted}
\paragraph{}
如果我们知道每个tuple元素都包含两个元素,for循环又可以进一步简写为:
\begin{minted}{python}
for index, name in enumerate(L):
    print index, '-', name
\end{minted}
\paragraph{}
可见,索引迭代也不是真的按索引访问,而是由 enumerate() 函数自动把每个元素变成 (index, element) 这样的tuple,再迭代,就同时获得了索引和元素本身。
\paragraph{}
zip()函数可以把两个 list 变成一个 list:
\begin{minted}{python}
>>> zip([10, 20, 30], ['A', 'B', 'C'])
[(10, 'A'), (20, 'B'), (30, 'C')]
\end{minted}
\section{迭代dict的value}
\paragraph{}
dict 对象有一个 values() 方法,这个方法把dict转换成一个包含所有value的list,这样,我们迭代的就是 dict的每一个 value:
\begin{minted}{python}
d = { 'Adam': 95, 'Lisa': 85, 'Bart': 59 }
print d.values()
# [85, 95, 59]
for v in d.values():
    print v
# 85
# 95
# 59
\end{minted}
\paragraph{}
dict除了values()方法外,还有一个 itervalues() 方法,用 itervalues() 方法替代 values() 方法,迭代效果完全一样:
\begin{minted}{python}
d = { 'Adam': 95, 'Lisa': 85, 'Bart': 59 }
print d.itervalues()
# <dictionary-valueiterator object at 0x106adbb50>
for v in d.itervalues():
    print v
# 85
# 95
# 59
\end{minted}
\section{迭代dict中的key和value}
\paragraph{}
dict 对象的 items() 方法返回的值:
\begin{minted}{python}
>>> d = { 'Adam': 95, 'Lisa': 85, 'Bart': 59 }
>>> print d.items()
[('Lisa', 85), ('Adam', 95), ('Bart', 59)]
\end{minted}
\paragraph{}
可以看到,items() 方法把dict对象转换成了包含tuple的list,我们对这个list进行迭代,可以同时获得key和value:
\begin{minted}{python}
>>> for key, value in d.items():
...     print key, ':', value
... 
Lisa : 85
Adam : 95
Bart : 59
\end{minted}
\end{CJK}
\end{document} 