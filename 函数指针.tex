\documentclass{article}
\usepackage{CJKutf8}
\usepackage{minted}
\begin{document}
\begin{CJK}{UTF8}{gbsn}
\title{函数指针}
\date{}
\maketitle
\section{基本概念}
\paragraph{}
程序运行期间,每个函数都会占用 一段连续的内存空间。而函数名就是该函数所占内存区域的起始地址(也称“入口地址”)。我们可以将函数的入口地址赋给一个指针变量,使该指针变量指向该函数。然后通过指针变量就可以调用这个函数。这种指向函数的指针变量称为“函数指针”。
\paragraph{定义形式}
类型名(*指针变量名)(参数类型1,参数类型2,...);
\subparagraph{例如:}
int(*pf)(int,char);
\subparagraph{}
表示pf是一个函数指针,它所指向的函数,返回值类型应该是int,该函数应有两个参数,第一个是int类型,第二个是char类型。
\paragraph{使用方法}
可以用一个原型匹配的函数的名字给一个函数指针赋值。要通过函数指针调用它所指向的函数,写法为:
\subparagraph{}
函数指针名(实参表); 
\begin{minted}{c++}
#include<iostream>
void PrintMin(int a,int b)
{
	if(a<b)
	printf("%d",a);
	else
	printf("%d",b);
}
int main()
{
	void(*pf)(int,int);
	int x=4,y=5;
	pf=PrintMin;
	pf(x,y);
	return 0;
}
\end{minted}
\section{函数指针和qsort库函数}
\paragraph{c语言快速排序库函数:}
\begin{minted}{c++}
void qsort(void*base,int nelem,unsigned int width,

int(*pfCompare)(const void*,const void*));
\end{minted}
\subparagraph{}
可以对任意类型的数组进行排序
\subparagraph{}
base:待排序数组的起始地址
\subparagraph{}
nelem:待排序数组的元素个数
\subparagraph{}
width:待排序数组的每个元素的大小(以字节为单位)
\subparagraph{}
pfcompare:是一个函数指针,指向一个比较函数,作用就是告诉qsort该怎么排序。
\section{实例}
\begin{minted}{c++}
#include<stdio.h>
#include<stdlib.h>
int MyCompare(const void * elem1,const void * elem 2)
{
	unsigned int * p1,* p2;
	p1=(unsigned int *)elem1;//"* elem1"非法,所以进行强制类型转换
	p2=(unsigned int *)elem2;//"* elem2"非法
	return (* p1%10)-(* p2%10);
}
#define NUM 5
int main()
{
	unsigned int an[NUM]={8,123,11,10,4};
	qsort(an,NUM,sizeof(unsigned int),MyCompare);//函数的名字和函数指针的类型是匹配的
	for(int i=0;i<NUM;i++)
	printf("%d",an[i]);
	return 0;
}
\end{minted}
\end{CJK}
\end{document} 