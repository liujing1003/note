\documentclass{article}
\usepackage{CJKutf8}
\usepackage{minted}
\usepackage[boxed]{algorithm2e}
\begin{document}
\begin{CJK}{UTF8}{gbsn}

\title{机器学习2}
\date{}
\maketitle
\section{Multiple Features}
\paragraph{}

Note:[$7:25-\theta^{T}$is a 1 by (n+1) matrix and not an (n+1) by 1 matrix]

\paragraph{}
Linear regression with multiple variables is also known as "multivariate linear regression".
\paragraph{}
We now introduce notation for equations where we can have any number of input variables.
\paragraph{}
\begin{algorithm}  
\paragraph{} 
$ {x}_j^{(i)} $=value of feature j in the $i^{th}$ training example
\paragraph{}
$x^{(i)}$=the input (features) of the $i^{th}$ training example
\paragraph{}
m=the number of training examples
\paragraph{}
n=the number of features
\end{algorithm} 
\paragraph{}
The multivariable form of the hypothesis function accommodating these multiple features is as follows:
\paragraph{}
$h_{\theta}(x)=\theta_{0}+\theta_{1}x_{1}+\theta_{2}x_{2}+\theta_{3}x_{3}+\cdots+\theta_{n}x_{n}$
\paragraph{}
In order to develop intuition about this function, we can think about $\theta_{0}$ as the basic price of a house, $\theta_{1}$ as the price per square meter, $\theta_{2}$ as the price per floor, etc. $x_{1} $will be the number of square meters in the house, $x_{2}$ the number of floors, etc.
\paragraph{}
Using the definition of matrix multiplication, our multivariable hypothesis function can be concisely represented as:
\paragraph{}
\begin{algorithm} 
\begin{eqnarray*}
h_{\theta}(x)=\left[\begin{array}{cccc}
\theta_{0}&\theta_{1}&\cdots&\theta_{n}
\end{array}\right]*\left[\begin{array}{c}
x_{0}\\
x_{1}\\
\vdots\\
x_{n}
\end{array}\right]=\theta^{T}x
\end{eqnarray*}
\end{algorithm}
\paragraph{}
This is a vectorization of our hypothesis function for one training example; see the lessons on vectorization to learn more.
\paragraph{}
Remark: Note that for convenience reasons in this course we assume $ x_{0}^{(i)}$ for (i∈1,…,m). This allows us to do matrix operations with theta and x. Hence making the two vectors $'\theta'$ and $ x^{(i)} $match each other element-wise (that is, have the same number of elements: n+1).
\end{CJK}
\end{document}