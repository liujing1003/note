\documentclass{article}
\usepackage{CJKutf8}
\usepackage{minted}
\begin{document}
\begin{CJK}{UTF8}{gbsn}
\title{函数对象}
\date{}
\maketitle
\section{定义}
\paragraph{若一个类重载了运算符"()",则该类的对象就成为函数对象}
\begin{minted}{c++}
class CMyAverage{//函数对象类
public:
	double operator()(int a1,int a2,int a3)
	{
		return (double)(a1+a2+a3)/3;
	}
};
	CMyAverage average;//函数对象
	cout<<average(3,2,3);//average.operator()(3,2,3)
	//输出:2.66667
\end{minted}
\section{函数对象的应用}
\paragraph{Accumulate源代码1}
\begin{minted}{c++}
template<typename_Inputlterator,typename_Tp>
_Tp accumulate(_Inputlterator__first,_Inputlterator__last,_Tp__init)
{
	for(;__first!=__last;++__first)
		__init=__init+*__first;
	return__init;
}//typename等价于class
\end{minted}
\paragraph{源代码2}
\begin{minted}{c++}
template<typename_Inputlterator,typename_Tp,typename_BinaryOperation>
  _Tp  accumulate(_Inputlterator__first,_Inputlterator__last,_Tp__init,  _BinaryOperation__binary_op)
  {
  	for(;__first!=__last;++__first)
  	      __nint=__binary_op(__init,*__first);
  	    return__init;
  }
  //调用accumulate时,和__binary_op对应的实参可以是个函数或函数对象
\end{minted}
\section{函数对象的应用示例}
\begin{minted}{c++}
#include<iostream>
#include<vector>
#include<algorithm>
#include<numeric>
#include<functional>
using namespace std;
int sumSquares(int total,int value)
{
	return total+value*value;
}
template<class T>
void Printinterval(T first,T last)
{//输出区间[first,last)中的元素
	for(;first!=last;++first)
		cout<<*first<<" ";
		cout<<endl;
}
template<class T>
class SumPowers
{
	private:
		int power;
	public:
	    SumPowers(int p):power(p){}
	    const T operator()(const T&total,const T&value)
	    {//计算value的power次方,加到total上
	    	T v=value;
	    	for(int i=0;i<power-1;++i)
	    		v=v*value;
	    	return total+v;
	    }
};
int main()
{
	const int SIZE=10;
	int a1[]={1,2,3,4,5,6,7,8,9,10};
	vector<int>v(a1,a1+SIZE);
	cout<<"1)";PrintInterval(v.begin(),v.end());
	int result=accumulate(v.begin(),v.end(),0,SumSquares);
	cout<<"2)平方和:"<<result<<endl;
	result=accumulate(v.begin(),v.end(),0,SumPowers<int>(3));
	cout<<"3)立方和:"<<result<<endl;
	result=accumulate(v.bagin(),v.end(),0,SumPowers<int>(4));
	cout<<"4)4次方和:"<<result;
	return 0;
} 
//输出:
//1)1 2 3 4 5 6 7 8 9 10
//2)平方和:385
//3)立方和:3025
//4)4次方和:25333
\end{minted}
\section{greater的应用}
\begin{minted}{c++}
#include<list>
#include<iosteam>
using namespace std;
class Myless
{
	public:
	bool operator()(const int&c1,const int&c2)
	{
		return (c1 % 10)<(c2 % 10);
	}
};
template<class T>
void Print(T first,T last)
{
	for(;first!=last;++first)
	cout<<*first<<",";
}
int main()
{
	const int SIZE=5;
	int a[SIZE]={5,21,14,2,3};
	list<int>lst(a,a+SIZE);
	lst.sort(MyLess());
	Print(lst.begin(),lst.end());
	cout<<endl;
	lst.sort(greater<int>());//greater<int>()是个对象
	Print(lst.begin(),lst.end());
	cout<<endl;
	return 0;
}
//输出:21,2,3,14,5
//    21,14,5,3,2
\end{minted}
\section{在STL中使用自定义的“大”,“小”关系}
\paragraph{} 
关联容器和STL中许多算法,都是可以用函数或函数对象自定义比较器的。在自定义了比较器op的情况下,以下三种说法是等价的:
\subparagraph{}
1)x小于y
\subparagraph{}
2)op(x,y)返回值为true
\subparagraph{}
3)y大于x
\section{例题}
\paragraph{写出MyMax模板}
\begin{minted}{c++}
#include<iostream>
#include<iterator>
using namespace std;
class MyLess
{
	public:
		bool operator()(int a1,int a2)
		{
		if((a1%10)<(a2%10))
			return true;
		else
			return false;
		}
		bool MyCompare(int a1,int a2)
		{
		if((a1%10)<(a2%10))
			return false;
		else
			return true;
};
template<class T,class Pred>
T MyMax(T first,T last,Pred myless)
{
	T tmpMax=first;
	for(;first!=last;++first)
	if(myless(*tmpMax,*first))
	tmpMax=first;
	return tmpMax;
};
int main()
{
	int a[]={35,7,13,19,12};
	cout<<*MyMax(a,a+5,MyLess())<<endl;
	cout<<*MyMax(a,a+5,MyCompare)<<endl;
	return 0;
}
//输出:19 12
\end{minted}
\end{CJK}
\end{document} 