\documentclass{article}
\usepackage{CJKutf8}
\usepackage{minted}
\begin{document}
\begin{CJK}{UTF8}{gbsn}\title{Python}
\date{}
\maketitle
\section{添加新元素}
\paragraph{}
append()总是把新的元素添加到list的尾部。
\paragraph{}
L.insert(0,'Paul')的意思是'Paul'将被添加到索引为0的位置上,其他同学自动向后移一位。
\section{删除元素}
\paragraph{}
pop()方法总是删掉list的最后一个元素,并且它还返回这个元素,所以我们执行L.pop()后,会打印出删掉的元素。
\paragraph{}
L.pop(n)把第n+1个元素删掉。
\section{替换元素}
\paragraph{}
L[2]='Paul'和L[-1]='Paul'都可以
\section{创建tuple}
\paragraph{}
tuple一旦创建就不能修改了,创建时与list唯一差别是用()代替[]。
\section{创建单元素tuple}
\paragraph{}
空tuple:t=()打印出()
\paragraph{}
t=(1)打印出1:因为()既可以表示tuple,又可以作为括号表示运算时的优先级。
\paragraph{}
t=(1,)打印出(1,)
\section{可变的tuple}
\paragraph{}
tuple所谓的不变是说tuple的每个元素指向永远不变,指向一个list就不能改成指向其他对象,但指向的这个list本身可变。
\section{if语句}
\paragraph{}
缩进规则:具有相同缩进的代码被视为代码块。
\paragraph{}
4个空格,不要使用tab
\paragraph{}
if后接表达式,然后用:表示代码块开始。
\paragraph{}
Python交互环境下,推出缩进需要多敲一行回车。
\section{if-else}
\paragraph{}
利用if-else语句,我们可以根据条件表达式的值为true或false分别执行if代码块或者else代码块。
\section{if-elif-else}
这一系列的条件判断会从上到下依次判断,如果某个判断为true,执行完对应的代码块,后面的条件判断就直接忽略,不再执行了。
\section{for循环}
\paragraph{}
\begin{minted}{Python}
L=['Adam','Lisa','Bart']
for name in L:
print name
\end{minted}
\paragraph{}
name这个变量是在for循环中定义的,意思是,依次取出list中的每一个元素,并把元素赋值给name,然后执行for循环体(就是缩进的代码块)
\section{while循环}
\paragraph{}
while循环不会迭代list或tuple的元素,而是根据表达式判断循环是否结束。
\paragraph{}
while循环每次先判断条件是否满足,如果为true,则执行循环体的代码块,否则,退出循环。
\section{break退出死循环}
\paragraph{}
用for循环或者while循环时,如果要在循环体内直接退出循环,可以使用break语句。可以在死循环内设if语句,满足时break即可退出循环。
\section{continue继续循环}
\paragraph{}
continue可以跳过后续循环代码,继续下一次循环。
\begin{minted}{python}
L=[75,98,59,81,66,43,69,85]
sum=0.0
n=0
for x in L:
    if x<60:
        continue
    sum=sum+x
    n=n+1
print sum/n
\end{minted}
\section{多重循环}
\paragraph{}
\begin{minted}{python}
for x in ['A','B','C']:
    for y in ['1','2','3']:
        print x+y
\end{minted}
\paragraph{}
x每循环一次,y就会循环3次。可以利用多重循环打印出全排列。        
\section{什么是dict}
\paragraph{}
组成查找表
\begin{minted}{python}
d={
    'Adam':95,
    'Lisa':85,
    'Bart':59
}
\end{minted}
\paragraph{}
我们把名字称为key,对应的成绩称为value,dic就是通过key来查找value;{}表示这是一个dic,然后按照key:value,写出来即可。最后一个key:value的逗号可以省略。
\paragraph{}
由于dic也是集合,len()函数可以计算任意集合的大小len(d)
\section{访问dict}
\paragraph{}
可以简单的使用d[key]的形式来查找对应的value,这和list很像,不同之处是,list必须使用索引返回对应的元素,而dict使用key.
\paragraph{}
一是判断key是否存在
\begin{minted}{python}
if 'Paul' in d:
    print d['Paul']
\end{minted}
\paragraph{}
如果'Paul'不存在,if语句判断为False,自然不会执行print,从而避免了错误。
\paragraph{}
二是使用dict本身提供的一个get方法,在key不存在的时候,返回none
\begin{minted}{python}
print d.get('Bert')
\end{minted}
\section{dict的特点}
key不能重复;存储的key-value序对是没有顺序的;key的元素必须不可变。
\begin{minted}{python}
{
    '123':[1,2,3],# key是str,value是list
    123:'123',# key是int,value是str
    ('a','b'):True #key是tuple,并且tuple的每个元素都是不可变对象,value是boolean
}
\end{minted}
\section{更新dict}
\paragraph{}
dict是可变的,也就是说,我们可以随时往dict中添加新的key-value;如果key已经存在,则赋值会用新的value值替换掉原来的value
\section{遍历dict}
\paragraph{}
由于dict是一个集合,所以,遍历dict和遍历list类似,都可以通过for循环实现。
\paragraph{}
直接使用for循环可以遍历dict的key:
\begin{minted}{python}
d={'Adam':95,'Lisa':85,'Bert':59}
for key in d:
   print key
\end{minted}
\paragraph{}
由于通过key可以获取对应的value,因此,在循环体内,可以获取到value的值。
\section{set}
\paragraph{}
dict的作用就是建立一组key和一组value的映射关系,dict的key是不能重复的。
\paragraph{}
set持有一系列元素,这一点和list很像,但是set的元素没有重复,而且是无序的,这点和dict的key很像。
\paragraph{}
创建set的方式是调用set()并传入一个list,list的元素将作为set的元素:
\begin{minted}{python}
s=set(['A','B','C'])
\end{minted}
\paragraph{}
可以查看set的内容
\begin{minted}{python}
print s
set(['A','B','C'])
\end{minted}
\paragraph{}
因为set不能包含重复的元素,所以,当我们传入包含重复元素的list,set会自动去掉重复的元素。
\section{访问set}
\paragraph{}
由于set存储的是无需集合,所以我们没法通过索引来访问。访问set中的某个成员实际上就是判断一个元素是否在set中。
\paragraph{}
我们可以用in操作符判断,大小写不同会被认为是两个不同的元素。
\section{set的特点}
\paragraph{}
set的内部结构和dict很像,唯一区别是不存储value,因此,判断一个元素是否在set中速度很快。
\paragraph{}
set存储的元素和dict的key类似,必须是不变对象,因此,任何可变对象是不能当如set中的。
\paragraph{}
最后,set存储的元素也是没有顺序的。
\begin{minted}{python}
weekdays=set(['MON','TUE','WED','THU','FRI','SAT',
'SUN'])
#再判断输入是否有效,只需要判断该字符串是否在set中:
x='???'
if x in weekdays:
   print 'input ok'
else:
   print 'input error'
\end{minted}
\section{遍历set}
\paragraph{}
元素的顺序和list的顺序很可能是不同的
\begin{minted}{python}
s=set(['Adam','Lisa','Bart'])
for name in s:
   print name
\end{minted}
\section{更新set}
\paragraph{}
一是把新的元素添加到set中,二是把已有元素从set中删除
\paragraph{}
添加元素时,用set的add()方法:
\begin{minted}{python}
s=set([1,2,3])
s.add(4)
print s
set([1,2,3,4])
\end{minted}
\paragraph{}
如果添加的元素已经有了,不会报错,但是不会加进去了
\paragraph{}
删除set中的元素时,用set的remove()方法:
\begin{minted}{python}
s=set([1,2,3,4])
s=remove(4)
print s
set([1,2,3])
\end{minted}
\paragraph{}
如果删除的元素不存在set中,remove()会报错
\end{CJK}
\end{document} 